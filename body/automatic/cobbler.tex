\chapter{Cobbler自动化装机}
\label{chap:cobbler}

\section{Cobbler 介绍}

号称补鞋匠都可以使用,大大降低了使用者的门槛,你懂的。

\section{Cobbler 安装及配置}

\begin{verbatim}
# yum install -y httpd dhcp tftp cobbler cobbler-web

# service httpd start
# service cobblerd start
# cobbler check # 检查是否通过,根据不通过的提示,进行修改配置文件
# vim /etc/cobbler/settings
  server: xxx.yyy.zzz.vvv
  next_server: xxx.yyy.zzz.vvv
# /etc/init.d/xinetd restart
# yum install -y pykickstart
# cobbler get-loaders
# openssl passwd -1 -salt 'lavenliu' 'laven'
fjsakjfsdkjfkdseri
# service cobblerd restart
# cobblerd check
# 如果检查没有问题,可以执行:
# cobbler sync
# vim /etc/cobbler/dhcp.template
#+END_EXAMPLE

配置完毕,接下来设置镜像:
BEGIN_EXAMPLE
mount /dev/cdrom /mnt
cobbler import --path=/mnt --name=CentOS-6.5-x86_64 --arch=x86_64
cobbler profile report # 可以找到默认ks配置文件
准备自定义的ks文件
cd /var/lib/cobbler/kickstarts
cobbler profile edit --name=CentOS-6.5-x86_64 --kickstart=/var/lib/cobbler/kickstarts/CentOS-6.5.x86_64.ks
cobbler profile report
每次改动,要执行sync动作
cobbler sync
  
镜像文件被复制到了/var/www/cobbler/ks_mirror/CentOS-6.5-x86_64目录下。
\end{verbatim}

%%% Local Variables:
%%% mode: latex
%%% TeX-master: t
%%% End:
