\section{tar命令的使用}
\label{sec:tarCmd}

tar\index{tar}命令是一个打包与解包的一个工具,功能很强大。下面介绍一些常用选项及使
用示例。

通用选项:

\begin{table}[htbp]
  \centering
    \caption{tar通用选项}
    \label{tab:tarGeneralOpt}
    \begin{tabular}{cl}
      \toprule
      选项     & 说明 \\
      \midrule
      j        & 使用bzip2的压缩方式 \\
      t        & 列出压缩包里有哪些文件,并不解压 \\
      z        & 使用gzip的压缩方式 \\
      f        & 指定输出的结果文件。该选项是必选的,不管是压缩还是解压缩 \\
      p        & 保留文件的所有权限 \\
      v        & 压缩或解压缩时,查看其打包过程 \\
      \bottomrule
    \end{tabular}
\end{table}

压缩时用的选项:

\begin{table}[!htbp]
  \centering
    \caption{tar压缩选项}
    \label{tab:tarCompressOpt}
    \begin{tabular}{cl}
      \toprule
      选项     & 说明 \\
      \midrule
      c        & 打包时用的选项,选项c与x不能同时出现 \\
      \bottomrule
    \end{tabular}
\end{table}

解压缩用的选项:

\begin{table}[htbp]
  \centering
    \caption{tar解压缩选项}
    \label{tab:tarUncompressOpt}
    \begin{tabular}{cl}
      \toprule
      选项     & 说明 \\
      \midrule
      x        & 解包时用的选项,选项c与x不能同时出现 \\
      \bottomrule
    \end{tabular}
\end{table}

举例说明:

\small{
\begin{verbatim}
  [root@iLiuc ~]# ls
  360fy        CLEAR-VOD-INSTALLPACKGE.V.2.0.8.tar
  这里有两个文件,第一个为目录,第二个为压缩包

  1. 创建tar包,不压缩
  [root@iLiuc ~]# tar -cvf 360fy.tar 360fy
  360fy/
  360fy/clearVodMS_360fy.tar.gz
  360fy/vod_yuezizhongxin.tar.gz
  360fy/clear_360fy.sql
  上面的例子我们使用-v选项,使我们可以看到过程。其中360fy.tar是我们创建的tar包,是针对
  360fy这个目录的,后面可以是一个或多个文件。

  2. 创建tar包,以gzip方式压缩
  [root@iLiuc ~]# tar -czvf 360fy.tar.gz 360fy

  3. 创建tar包,以bzip2的方式压缩
  [root@iLiuc ~]# tar -cjvf 360fy.tar.bz2 360fy

  4. 查看压缩包的内容,并不解压缩
  [root@iLiuc ~]# tar -tf 360fy.tar
  360fy/
  360fy/clearVodMS_360fy.tar.gz
  360fy/vod_yuezizhongxin.tar.gz
  360fy/clear_360fy.sql

  5. 解压缩
  不管是tar包,还是以gzip或bz2压缩的方式,我们使用一下这条命令都是通用的
  [root@iLiuc ~]# tar -xf 360fy.tar
  [root@iLiuc ~]# tar -xf 360fy.tar.gz
  [root@iLiuc ~]# tar -xf 360fy.tar.bz2
  这里我们没有加-v选项,可以加上-v选项以看到解压过程
\end{verbatim}
}
\normalsize
