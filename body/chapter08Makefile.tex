\chapter{Makefile文件}
\label{sec:makefile}

一个工程中的源文件不计数,其按类型、功能、模块分别放在若干个目录中,
makefile定义了一系列的规则来指定,哪些文件需要先编译,哪些文件需要后编
译,哪些文件需要重新编译,甚至于进行更复杂的功能操作,因为makefile就像
一个Shell脚本一样,其中也可以执行操作系统的命令。

\section{定义和概述}

Linux环境下的程序员如果不会使用GNU make来构建和管理自己的工程,应该不能
算是一个合格的专业程序员,至少不能称得上是Unix程序员。在 Linux(unix)
环境下使用GNU的make工具能够比较容易的构建一个属于你自己的工程,整个工程
的编译只需要一个命令就可以完成编译、连接以至于最后的执行。不过这需要我
们投入一些时间去完成一个或者多个称之为Makefile文件的编写。

所要完成的Makefile文件描述了整个工程的编译、连接等规则。其中包括:工程
中的哪些源文件需要编译以及如何编译、需要创建那些库文件以及如何创建这些
库文件、如何最后产生我们想要的可执行文件。尽管看起来可能是很复杂的事情,
但是为工程编写Makefile的好处是能够使用一行命令来完成“自动化编译”,一旦
提供一个(通常对于一个工程来说会是多个)正确的Makefile。编译整个工程你
所要做的唯一的一件事就是在shell提示符下输入make命令。整个工程完全自动编
译,极大提高了效率。

make是一个命令工具,它解释Makefile中的指令(应该说是规则)。在Makefile
文件中描述了整个工程所有文件的编译顺序、编译规则。Makefile有自己的书写
格式、关键字、函数。像C语言有自己的格式、关键字和函数一样。而且在
Makefile中可以使用系统shell所提供的任何命令来完成想要的工作。
Makefile(在其它的系统上可能是另外的文件名)在绝大多数的IDE开发环境中都
在使用,已经成为一种工程的编译方法。

\section{自动化编译}

makefile带来的好处就是\--“自动化编译”,一旦写好,只需要一个make命令,整
个工程完全自动编译,极大的提高了软件开发的效率。make是一个命令工具,是
一个解释makefile中指令的命令工具,一般来说,大多数的IDE都有这个命令,比
如:Delphi的make,Visual C++的nmake,Linux下GNU的make。可见,makefile都
成为了一种在工程方面的编译方法。

\section{主要功能}

Make工具最主要也是最基本的功能就是通过makefile文件来描述源程序之间的相
互关系并自动维护编译工作。而makefile文件需要按照某种语法进行编写,文件
中需要说明如何编译各个源文件并连接生成可执行文件,并要求定义源文件之间
的依赖关系。makefile文件是许多编译器--包括 Windows NT下的编译器--维护编
译信息的常用方法,只是在集成开发环境中,用户通过友好的界面修改
makefile文件而已。

在UNIX系统中,习惯使用Makefile作为makefile文件。如果要使用其他文件作为
makefile,则可利用类似下面的make命令选项指定makefile文件:

\subsection{make命令}

\subsection{Makefile的规则}

\subsection{文件定义与命令}

\subsection{有效的宏引用}

\subsection{预定义变量}

\section{给个实例看看}
