\chapter{LAMP}

\section{安装依赖包}

Needless to say, you should install required system packages. So you
cancontinue the next steps. :-)

Be sure these packages are installed:

\small{
\begin{verbatim}
  gcc
  gcc-c++
  flex
  bison
  autoconf
  automake
  bzip2-devel
  ncurses-devel
  libjpeg-devel
  libpng-devel
  libtiff-devel
  freetype-devel
  pam-devel
\end{verbatim}
}
\normalsize

\section{安装额外的包}

In this section, we need to compile and install four packages:

\begin{enumerate}[itemsep=0pt,parsep=0pt]
\item GD2
  \small{
\begin{verbatim}
  [root@lamp ~] # cd /usr/local/src
  [root@lamp src] # tar xzvf gd-2.0.34.tar.gz
  [root@lamp src] # cd gd-2.0.34
  [root@lamp gd-2.0.34] # ./configure --prefix=/usr/local/gd2
  [root@lamp gd-2.0.34] # make
  [root@lamp gd-2.0.34] # make install  
\end{verbatim}
  }
  \normalsize

\item LibXML2
  \small{
\begin{verbatim}
  [root@lamp ~] # cd /usr/local/src
  [root@lamp src] # tar xzvf libxml2-2.6.29.tar.gz
  [root@lamp src] # cd libxml2-2.6.29
  [root@lamp libxml2-2.6.29] # ./configure --prefix=/usr/local/libxml2
  [root@lamp libxml2-2.6.29] # make
  [root@lamp libxml2-2.6.29] # make install
\end{verbatim}
  }
  \normalsize

\item LibMcrypt
  \small{
\begin{verbatim}
  [root@lamp ~] # cd /usr/local/src
  [root@lamp src] # tar xjvf libmcrypt-2.5.8.tar.bz2
  [root@lamp src] # cd libmcrypt-2.5.8
  [root@lamp libmcrypt-2.5.8] # ./configure --prefix=/usr/local/libmcrypt
  [root@lamp libmcrypt-2.5.8] # make
  [root@lamp libmcrypt-2.5.8] # make install
\end{verbatim}
  }
  \normalsize

\item OpenSSL
  \small{
\begin{verbatim}
  [root@lamp ~] # cd /usr/local/src
  [root@lamp src] # tar xzvf openssl-0.9.8e.tar.gz
  [root@lamp src] # cd openssl-0.9.8e
  [root@lamp openssl-0.9.8e] # ./config --prefix=/usr/local/openssl
  [root@lamp openssl-0.9.8e] # make
  [root@lamp openssl-0.9.8e] # make test
  [root@lamp openssl-0.9.8e] # make install
\end{verbatim}
  }
  \normalsize

\end{enumerate}

\section{编译安装MySQL}

\small{
\begin{verbatim}
  [root@lamp ~] #tar xzvf mysql-5.0.27.tar.gz
  [root@lamp ~] # cd mysql-5.0.27
  [root@lamp mysql-5.0.27] # ./configure \
  "--prefix=/usr/local/mysql" \
  "--localstatedir=/var/lib/mysql" \
  "--with-mysqld-user=mysql" \
  "--without-debug" \
  "--with-big-tables" \
  "--with-extra-charsets=all" \
  "--with-pthread" \
  "--enable-static" \
  "--enable-thread-safe-client" \
  "--with-client-ldflags=-all-static" \
  "--with-mysqld-ldflags=-all-static" \
  "--enable-assembler" \
  "--without-isam" \
  "--without-innodb" \
  "--without-ndb-debug"
  
  [root@lamp msyql-5.0.27] # make
  [root@lamp mysql-5.0.27] # make install
  [root@lamp mysql-5.0.27] # useradd mysql
  [root@lamp mysql-5.0.27] # cd /usr/local/mysql
  [root@lamp mysql] # bin/mysql_install_db --user=mysql
  [root@lamp mysql] # chown -R root:mysql .
  [root@lamp mysql] # chown -R mysql /var/lib/mysql
  [root@lamp mysql] # cp share/mysql/my-huge.cnf /etc/my.cnf
  [root@lamp mysql] # cp share/mysql/mysql.server /etc/rc.d/init.d/mysqld
  [root@lamp mysql] # chmod 755 /etc/rc.d/init.d/mysqld
  [root@lamp mysql] # chkconfig --add mysqld
  [root@lamp mysql] # chkconfig --level 3 mysqld on
  [root@lamp mysql] # /etc/rc.d/init.d/mysqld start
  [root@lamp mysql] # bin/mysqladmin -u root password 'clear123'
\end{verbatim}
}
\normalsize

\section{编译安装Apache}

\small{
\begin{verbatim}
  [root@lamp src] # cd /usr/local/src
  [root@lamp src] # tar xjvf httpd-2.2.4.tar.bz2
  [root@lamp src] # cd httpd-2.2.4
  [root@lamp httpd-2.2.4] # ./configure \
  "--prefix=/usr/local/apache2" \
  "--with-included-apr" \
  "--enable-so" \
  "--enable-deflate=shared" \
  "--enable-expires=shared" \
  "--enable-rewrite=shared" \
  "--enable-static-support" \
  "--disable-userdir"
  [root@lamp httpd2.2.4] # make
  [root@lamp httpd2.2.4] # make install
  [root@lamp httpd2.2.4] # echo '/usr/local/apache2/bin/apachectl start' \
  > >> /etc/rc.local
\end{verbatim}
}
\normalsize

\section{编译安装PHP}

\small{
\begin{verbatim}
  [root@lamp ~] # cd /usr/local/src
  [root@lamp src] # tar xjvf php-5.2.3.tar.bz2
  [root@lamp php-5.2.3] # cd php-5.2.3
  [root@lamp php-5.2.3] # ./configure \
  "--prefix=/usr/local/php" \
  "--with-apxs2=/usr/local/apache2/bin/apxs" \
  "--with-config-file-path=/usr/local/php/etc" \
  "--with-mysql=/usr/local/mysql" \
  "--with-libxml-dir=/usr/local/libxml2" \
  "--with-gd=/usr/local/gd2" \
  "--with-jpeg-dir" \
  "--with-png-dir" \
  "--with-bz2" \
  "--with-freetype-dir" \
  "--with-iconv-dir" \
  "--with-zlib-dir " \
  "--with-openssl=/usr/local/openssl" \
  "--with-mcrypt=/usr/local/libmcrypt" \
  "--enable-soap" \
  "--enable-gd-native-ttf" \
  "--enable-memory-limit" \
  "--enable-ftp" \
  "--enable-mbstring" \
  "--enable-exif" \
  "--disable-ipv6" \
  "--disable-cgi" \
  "--disable-cli"
  [root@lamp php-5.2.3] # make
  [root@lamp php-5.2.3] # make install
  [root@lamp php-5.2.3] # mkdir /usr/local/php/etc
  [root@lamp php-5.2.3] # cp php.ini-dist /usr/local/php/etc/php.ini
\end{verbatim}
}
\normalsize

\section{安装Zend加速器}

\small{
\begin{verbatim}
  [root@lamp ~]# cd /usr/local/src
  [root@lamp ~]# tar xf ZendOptimizer-3.2.8-linux-glibc21-i386.tar.gz
  [root@lamp ~]# ./ZendOptimizer-3.2.8-linux-glibc21-i386/install.sh
\end{verbatim}
}
\normalsize

\section{整合Apache与PHP}

We should modify configure file httpd.conf:

\small{
\begin{verbatim}
[root@lamp ~]# emacs /usr/local/apache2/conf/httpd.conf
\end{verbatim}
}
\normalsize

Find this line,

\small{
\begin{verbatim}
AddType application/x-gzip .gz .tgz
\end{verbatim}
}
\normalsize

Add on line under this line,

\small{
\begin{verbatim}
AddType application/x-httpd-php .php
\end{verbatim}
}
\normalsize

Find these lines,

\small{
\begin{verbatim}
<IfModule dir_module>
DirectoryIndex index.html
<IfModule>
\end{verbatim}
}
\normalsize

Change them like this,

\small{
\begin{verbatim}
<IfModule dir_module>
DirectoryIndex index.html index.htm index.php
<IfModule>
\end{verbatim}
}
\normalsize

Find these lines and uncomment them,

\small{
\begin{verbatim}
#Include conf/extra/httpd-mpm.conf
#Include conf/extra/httpd-info.conf
#Include conf/extra/httpd-vhosts.conf
#Include conf/extra/httpd-default.conf
\end{verbatim}
}
\normalsize

When finished, save it! Then restart apache service,

\small{
\begin{verbatim}
[root@lamp ~]# /usr/local/apache2/bin/apachectl restart
\end{verbatim}
}
\normalsize

%%% Local Variables:
%%% mode: latex
%%% TeX-master: t
%%% End:
