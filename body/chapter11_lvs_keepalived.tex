\chapter{LVS+Keepalived负载均衡集群}

LVS(Linux Virtual Server) is a cluster of servers 

The Linux Virtual Server can be used to build scalable network
services based on a cluster of two or more nodes. The active node of
the cluster redirects service requests to a collection of server hosts
that will actually perform the services. Supported features include
two protocols (TCP and UDP), three packet-forwarding methods (NAT,
tunneling, and direct routing), and eight load balancing algorithms
(round robin, weighted round robin, least-connection, weighted
least-connection, locality-based least-connection, locality-based
least-connection with replication, destination-hashing, and
source-hashing).

LVS: ipvsadm/ipvs

When a new connection is requested from a client to a service provided
by the LVS (e.g. httpd), the director will choose a realserver from
the client.

From then, all packets from the client will go through the director to
that particular realserver.

The association between the client and the realserver will last for
only the life of the tcp connection (or udp exchange).

For the next tcp connection, the director will choose a new realserver
(which may or may not be the same as the first realserver).

Thus a web browser connecting to a LVS serving a webpage consisting of
several hits (images, html page), may get each hit from a separate
realserver.

LVS IP Address Name Conventions

Virtual IP (VIP) address: The IP address the Director uses to offer
services to client computers

Real IP (RIP) address: The IP address used on the cluster nodes

Director's IP (DIP) address: The IP address the Director uses to
connect to the D/RIP network

Client computer's IP (CIP) address: The IP address assigned to a
client computer that it uses as a source IP address for requests sent
to the cluster

\begin{quote}
  Basic Properties of LVS-NAT 

1. The cluster nodes need to be on the
same network (VLAN or subnet) as the Director.
集群节点必须与调度器在同一个网络中


2. The RIP addresses of the cluster nodes are normally private,
non-routable IP addresses used only for intracluster communication.
RIP通常是私有地址,仅用于各集群节点间的通信

3. director位于client与real server之间,并负责处理进出的所有通信

4. realserver必须将网关指向DIP

5. 支持端口映射

6. real server可以使用任何操作系统

7. 较大规模应用场景中,director易成为系统瓶颈
\end{quote}

\begin{quote}
  1. 集群节点跟director必须在同一物理网络中

  2. RIP可以使用公网地址,实现便捷的远程管理和监控

  3. director仅负责处理入站请求,响应报文则由real server直接发往客户端

  4. real server不能讲网关指向DIP

  5. 不支持端口映射

  real server必须能够隐藏VIP
\end{quote}

\begin{quote}
  TUN:
  1. 集群节点可以跨越Internet
  2. RIP必须是公网地址
  3. Director仅处理入站请求,响应报文则由real server直接发往客户端
  4. real server网关不能指向director
  5. 只有支持隧道功能的os才能用于real server
  6. 不支持端口映射
\end{quote}

\section{LVS调度算法}

Director在接收到来自于Client的请求时,会基于"schedule"从RealServer中选
择一个响应给Client。ipvs支持以下调度算法:

\begin{enumerate}[itemsep=0pt,parsep=0pt]
\item 轮询(round robin, rr),加权轮询(Weighted round robin, wrr)

  新的连接请求被轮流分配至各RealServer;算法的优点是其简洁性,它无需记
  录当前所有连接的状态,所以它是一种无状态调度。轮叫调度算法假设所有服
  务器处理性能均相同,不管服务器的当前连接数和响应速度。该算法相对简单,
  不适用于服务器组中处理性能不一的情况,而且当请求服务时间变化比较大时,
  轮叫调度算法容易导致服务器间的负载不平衡。

\item 最少连接(least connected, lc), 加权最少连接(weighted least
  connection, wlc)

  新的连接请求将被分配至当前连接数最少的RealServer;最小连接调度是一种
  动态调度算法,它通过服务器当前所活跃的连接数来估计服务器的负载情况。
  调度器需要记录各个服务器已建立连接的数目,当一个请求被调度到某台服务
  器,其连接数加1;当连接中止或超时,其连接数减一。

\item 基于局部性的最少链接调度(Locality-Based Least Connections
  Scheduling,lblc)

  针对请求报文的目标IP地址的负载均衡调度,目前主要用于Cache集群系统,因
  为在Cache集群中客户请求报文的目标IP地址是变化的。这里假设任何后端服务
  器都可以处理任一请求,算法的设计目标是在服务器的负载基本平衡情况下,
  将相同目标IP地址的请求调度到同一台服务器,来提高各台服务器的访问局部
  性和主存Cache命中率,从而整个集群系统的处理能力。LBLC调度算法先根据请
  求的目标IP地址找出该目标IP地址最近使用的服务器,若该服务器是可用的且
  没有超载,将请求发送到该服务器;若服务器不存在,或者该服务器超载且有
  服务器处于其一半的工作负载,则用“最少链接”的原则选出一个可用的服务
  器,将请求发送到该服务器。

\item 带复制的基于局部性最少链接调度(Locality-Based Least Connections
  with Replication Scheduling,lblcr)

  也是针对目标IP地址的负载均衡,目前主要用于Cache集群系统。它与LBLC算法
  的不同之处是它要维护从一个目标IP地址到一组服务器的映射,而 LBLC算法维
  护从一个目标IP地址到一台服务器的映射。对于一个“热门”站点的服务请求,
  一台Cache 服务器可能会忙不过来处理这些请求。这时,LBLC调度算法会从所
  有的Cache服务器中按“最小连接”原则选出一台Cache服务器,映射该“热
  门”站点到这台Cache服务器,很快这台Cache服务器也会超载,就会重复上述
  过程选出新的Cache服务器。这样,可能会导致该“热门”站点的映像会出现在
  所有的Cache服务器上,降低了Cache服务器的使用效率。LBLCR调度算法将“热
  门”站点映射到一组Cache服务器(服务器集合),当该“热门”站点的请求负
  载增加时,会增加集合里的Cache服务器,来处理不断增长的负载;当该“热
  门”站点的请求负载降低时,会减少集合里的Cache服务器数目。这样,该“热
  门”站点的映像不太可能出现在所有的Cache服务器上,从而提供Cache集群系
  统的使用效率。LBLCR算法先根据请求的目标IP地址找出该目标IP地址对应的服
  务器组;按“最小连接”原则从该服务器组中选出一台服务器,若服务器没有
  超载,将请求发送到该服务器;若服务器超载;则按“最小连接”原则从整个
  集群中选出一台服务器,将该服务器加入到服务器组中,将请求发送到该服务
  器。同时,当该服务器组有一段时间没有被修改,将最忙的服务器从服务器组
  中删除,以降低复制的程度。

\item 目标地址散列调度(Destination Hashing,dh)

  算法也是针对目标IP地址的负载均衡,但它是一种静态映射算法,通过一个散
  列(Hash)函数将一个目标IP地址映射到一台服务器。目标地址散列调度算法
  先根据请求的目标IP地址,作为散列键(Hash Key)从静态分配的散列表找出
  对应的服务器,若该服务器是可用的且未超载,将请求发送到该服务器,否则
  返回空。

\item 源地址散列调度(Source Hashing,sh)

  算法正好与目标地址散列调度算法相反,它根据请求的源IP地址,作为散列键
  (Hash Key)从静态分配的散列表找出对应的服务器,若该服务器是可用的且
  未超载,将请求发送到该服务器,否则返回空。它采用的散列函数与目标地址
  散列调度算法的相同。除了将请求的目标IP地址换成请求的源IP地址外,它的
  算法流程与目标地址散列调度算法的基本相似。在实际应用中,源地址散列调
  度和目标地址散列调度可以结合使用在防火墙集群中,它们可以保证整个系统
  的唯一出入口。
\end{enumerate}


基于DNS的负载均衡方案性能可能会出现问题。DNS的记录会缓存。

rsync基于文件的同步,效率低。

drbd磁盘镜像,让两个计算机的两块磁盘做镜像,基于块级别的同步,效率高。

\section{安装LVS}
\label{installLVS}

\subsection{环境准备}
\label{sec:lvsEnvPrepare}

