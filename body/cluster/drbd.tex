\chapter{DRBD}
\label{chap:drbd}

distributed replicated block device

a software-based, shared-nothing, replicated storage solution
mirroring the content of block devices (hard disks, partitions,
logical volumes etc.) between servers.

DRBD's core functionality is implemented by way of a Linux Kernel
module.

User space administration tools

drbdadm

The high-level administration tool of the DRBD program suite. It
obtains all DRBD configuration parameters from the configuration 
file /etc/drbd.conf

drbdsetup

The program that allows users to configure the DRBD module that has
been loaded into the running kernel. It is the low-level tool within
the DRBD program suite.

Resource roles

In DRBD, every resource has a role, which may be Primary or Secondary.

Primary

A DRBD device in the primary role can be used unrestrictedly for read
and write operations. It may be used for creating and mounting file
systems, raw or direct I/O to the block device, etc.

Secondary

A DRBD device in the secondary role receives all updates from the perr
node's device, but otherwise disallows access completely.  It can not
be used by applications, neither for read nor write access.

DRBD modes

Single-primary mode
\begin{quote}
Any resource is, at any given time, in the primary role on only one
cluster member.

Only one cluster node manipulates the data at any moment.

This mode can be used with any conventional file system (ext3, ext4,
XFS, etc.)

Deploying DRBD in single-primary mode is the canonical approach for
high availability (fail-over capable) clusters.
\end{quote}

Dual-primary mode
\begin{quote}
Any resource is, at any given time, in the primary role on both
cluster nodes.

Since concurrent access to the data is thus possible, this mode
requires the use of a shared cluster file system that utilizes a
distributed lock manager. Examples include GFS and OCFS2.

Deploying DRBD in dual-primary mode is the preferred approach for
load-balancing clusters which require concurrent data access from two
nodes.

Disabled by default, and must be enabled explicitly in DRBD's
configuration file.

Available in DRBD 8.0 and later.
\end{quote}
 
\section{三种同步模式}

\begin{enumerate}[itemsep=0pt,parsep=0pt]
\item 协议A
\begin{quote}
异步复制协议。只要主节点完成本地写操作就认为写操作完成,并且需要复制的
数据包会被存储到本地TCP发送缓冲中。当发生failover故障,数据可能会丢失。
当发生failover故障时,在备节点的数据被认为仍是稳固的,然而,在crash发生
的时间点上很多最新的更新操作会丢失。
\end{quote}

\item 协议B
\begin{quote}
内存同步(半同步)复制协议。当本地盘的写已经完成,并且复制数据包已经到
达对应从节点,此时主节点才认为磁盘写已经完成。通常情况下,发生failover
不会导致数据丢失(因为后备系统内存中已经获得了数据更新)。然而,如果所
有节点同时出现电源故障,则主节点数据存储会产生不可逆的错误结构,主节点
上多数最新写入的数据可能会丢失。
\end{quote}

\item 协议C
\begin{quote}
同步复制协议。只有在本地和远程磁盘都确定写入已经完成时,主节点才会认为
写入完成。这样可确保发生故障时不会导致任何数据丢失。如果发生数据丢失的
现象,那也只会在所有节点同时存在错误存储时才会发生。
\end{quote}
\end{enumerate}

Efficient synchronization
\begin{quote}
Synchronization is necessary if the replication link has been
interrupted for any reason

1. failure of the primary node
2. failure of the secondary node
3. or interruption of the replication link

(Re-)synchronization is distinct from device replication
\end{quote}

Split brain
\begin{quote}
Split brain is a situation where, due to temporary failure of all
network links between cluster nodes, and possibly due to intervension
by a cluster management software or human error, both nodes switched
to the primary role while disconnected.
\end{quote}

automatic recovery
\begin{quote}
DRBD allows for automatic operator notification (by email or other
means) when it detects split brain.

DRBD has several resolution algorithms available for resolving the
split brain automaticly.

1. Discarding modifications made on the "younger" primary
2. Discarding modifications made on the "older" primary
3. Discarding modifications on the primary with fewer changes
4. Graceful recovery from split brain if one host has had no intermediate changes
\end{quote}

\section{安装及配置DRBD}

DRBD包含两大主要组件,内核空间中的驱动代码与用户空间中的管理工具。自从
Linux内核2.6.33版本及之后版本,DRBD的驱动代码已整合进了内核模块。所以,
在2.6.33版本及更新版本后,我们只需要安装用户空间的管理工具即可。否则,
我们需要同时安装内核支持模块与管理工具两个包,并且两个工具的版本号要保
持一致。

实验环境:

\begin{figure}[!h]
\centering
\begin{tabular}{llll}
\toprule
机器列表        & IP地址        & 操作系统       & DRBD版本 \\
\midrule
node1.laven.com & 192.168.1.15  & RHEL5U5 32bit  & 8.3.8 \\
node2.laven.com & 192.168.1.16  & RHEL5U5 32bit  & 8.3.8 \\
\bottomrule
\end{tabular}
\end{figure}


开始安装

\small{
\begin{verbatim}
[root@node1 ~]# ls
drbd83-8.3.8-1.el5.centos.i386.rpm
kmod-drbd83-8.3.8-1.el5.centos.i386.rpm

[root@node1 ~]# rpm -ivh kmod-drbd83-8.3.8-1.el5.centos.i386.rpm
[root@node1 ~]# rpm -ivh drbd83-8.3.8-1.el5.centos.i386.rpm
\end{verbatim}
}
\normalsize

在node2上进行同样的安装操作。

开始配置,

\small{
\begin{verbatim}
[root@node1 ~]# ls
drbd83-8.3.8-1.el5.centos.i386.rpm
kmod-drbd83-8.3.8-1.el5.centos.i386.rpm

[root@node1 ~]# rpm -ivh kmod-drbd83-8.3.8-1.el5.centos.i386.rpm
[root@node1 ~]# rpm -ivh drbd83-8.3.8-1.el5.centos.i386.rpm
\end{verbatim}
}
\normalsize

安装完成之后,来个简单的配置,使drbd可以简单的用起来。drbd的主配置文件
为/etc/drbd.conf,其中还有一个/etc/drbd.d目录,里面放置了其他的配置文件,
这也是便于管理,因此,我们可以分段进行配置,然后放置到该目录即可,在主
配置文件中使用include指令把这些文件加载进来即可。

看一下全局配置文件,

\small{
\begin{verbatim}
[root@node1 ~]# cat /etc/drbd.d/global_common.conf 
global {
   usage-count no;
   # minor-count dialog-refresh disable-ip-verification
}

common {
   protocol C;

   handlers {
      pri-on-incon-degr "/usr/lib/drbd/notify-pri-on-incon-degr.sh; \
      /usr/lib/drbd/notify-emergency-reboot.sh; \
      echo b > /proc/sysrq-trigger ; reboot -f";

      pri-lost-after-sb "/usr/lib/drbd/notify-pri-lost-after-sb.sh; \
      /usr/lib/drbd/notify-emergency-reboot.sh; \
      echo b > /proc/sysrq-trigger ; reboot -f";

      local-io-error "/usr/lib/drbd/notify-io-error.sh; \
      /usr/lib/drbd/notify-emergency-shutdown.sh; \
      echo o > /proc/sysrq-trigger ; halt -f";

      fence-peer "/usr/lib/drbd/crm-fence-peer.sh";
      split-brain "/usr/lib/drbd/notify-split-brain.sh root";
      out-of-sync "/usr/lib/drbd/notify-out-of-sync.sh root";
      before-resync-target "/usr/lib/drbd/snapshot-resync-target-lvm.sh \
      -p 15 -- -c 16k";
       after-resync-target /usr/lib/drbd/unsnapshot-resync-target-lvm.sh;
   }

   startup {
       wfc-timeout 120;
       degr-wfc-timeout 120;
   }

   disk {
       on-io-error detach;
       fencing resource-only;
   }

   net {
       cram-hmac-alg "sha1";
       shared-secret "mypass";
   }

   syncer {
       rate 100M;
   }
}
\end{verbatim}
}
\normalsize

现在定一个简单的资源r0,资源文件后缀以res结尾。

\small{
\begin{verbatim}
[root@node1 ~]# cat /etc/drbd.d/r0.res 
resource r0 {
  on node1.laven.com {
    device    /dev/drbd0;
    disk      /dev/sdb1;
    address   192.168.1.15:7789;
    meta-disk internal;
  }

  on node2.laven.com {
    device    /dev/drbd0;
    disk      /dev/sdb1;
    address   192.168.1.16:7789;
    meta-disk internal;
  }
}
\end{verbatim}
}
\normalsize

配置完毕,我们可以把这些文件复制到节点node2上去,

\small{
\begin{verbatim}
[root@node1 ~]# scp /etc/drbd.d/* node2:/etc/drbd.d/
\end{verbatim}
}
\normalsize

好了,初始化两端的drbd,然后启动drbd服务。

\small{
\begin{verbatim}
[root@node1 ~]# drbdadm create-md r0
[root@node2 ~]# drbdadm create-md r0
\end{verbatim}
}
\normalsize

初始化完毕之后,我们把node1作为主节点,node2作为从节点。之后,我们把
/dev/drbd0格式化然后挂载到/web目录下,然后在/web路下创建一些文件,然后
到node2上检查这些文件是否可以看到。

\small{
\begin{verbatim}
[root@node1 ~]# fdisk /dev/sdb -> /dev/sdb1
[root@node1 ~]# mke2fs -j /dev/drbd0
[root@node1 ~]# mount /dev/drbd0 /web
[root@node1 ~]# drbdadm primary r0
[root@node1 ~]# cd /web
[root@node1 web]# for i in {1..10}; do echo "Hello, world" > file${i}; \
> done
[root@node1 web]# ls
file1  file10  file2  file3  file4  file5  
file6  file7   file8  file9  lost+found
\end{verbatim}
}
\normalsize

这时,我们把node2作为主节点,并把/dev/drbd0挂载到/web目录下,验证是不是
有上述文件。首先,需要把/web目录umount,然后把node1变为从节点。之后,
node2需要先把自己变为主节点,然后挂载/dev/drbd0到/web下即可,

\small{
\begin{verbatim}
[root@node1 ~]# umount /web
[root@node1 ~]# drbdadm secondary r0
[root@node2 ~]# drbdadm primary r0
[root@node2 ~]# mount /dev/drbd0 /web
[root@node2 ~]# ls /web
file1  file10  file2  file3  file4  file5  
file6  file7   file8  file9  lost+found
[root@node2 ~]# cat /web/file*
Hello, world
Hello, world
Hello, world
Hello, world
Hello, world
Hello, world
Hello, world
Hello, world
Hello, world
Hello, world
\end{verbatim}
}
\normalsize