\chapter{Keepalived}
\label{chap:keepalived}

\section{Keepalived简介}

KeepAlived起初是专为LVS设计的,专门用来监控LVS集群系统中各个服务节点的
状态,后来又加入了VRRP的功能,因此,除了配合LVS服务外,也可以作为其他服
务(如Nginx,HAProxy等)的高可用软件。VRRP是Virtual Router Redundancy
Protocol(虚拟路由冗余协议)的缩写,VRRP出现的目的就是为了解决静态路由
出现的单点故障问题,它能够保证网络的不间断、稳定地运行。所
以,KeepAlived一方面具有LVS cluster nodes healthcheck功能,另一方面也具
有LVS directors failover功能。

\section{Keepalived安装部署}

\subsection{环境准备}

演示环境为CentOS6.5 64位系统,机器列表如下:

\begin{table}[!htbp]
  \centering
  \caption{KeepAlived演示环境机器列表}
  \begin{tabular}{|l|l|r|}
    \hline
    主机名  & IP地址 & 角色 \\
    \hline
    lb01.lavenliu.com & 192.168.20.150 & KeepAlived(主) \\
    \hline
    lb02.lavenliu.com & 192.168.20.151 & KeepAlived(备) \\
    \hline
  \end{tabular}
\end{table}

\subsection{开始安装}

\begin{verbatim}
yum install -y keepalived
\end{verbatim}

\subsection{Keepalived配置介绍}

\section{运行服务与故障模拟}
