\section{crontab命令的使用}
\label{sec:crontabCmd}

假想这样一个场景:每天的凌晨两点,领导都会要求你重启服务器\footnote{当然,这
有点变态}。这时,你该怎么办?你是不是每天凌晨两点都要从温暖的被窝里爬
出来:然后远程连接服务器、然后重启服务器、然后重新钻进被窝、然后失眠
了...。每天都如此,我想你一定会崩溃的。

crontab\index{crontab}命令可以解救你!crontab几个字段的说明:

\small{
\begin{verbatim}
  field          allowed values
  -----          --------------
  minute         0-59
  hour           0-23
  day of month   1-31
  month          1-12 (or names, see below)
  day of week    0-7 (0 or 7 is Sun, or use names)
\end{verbatim}
}
\normalsize

\small{
\begin{verbatim}
  # 查看当前用户的crontab
  [root@iLiuc ~]# crontab -l
  */2 * * * * /usr/lib/clear-server/cleargard/cleargard.sh
  上面语句的意思是,每2分钟去执行/usr/lib/clear-server/cleargard目录下的
  cleargard.sh脚本,只要系统一直运行,它就会循环往复的执行。

  # 编辑crontab
  [root@iLiuc ~]# crontab -e
\end{verbatim}
}
\normalsize

\section{crontab具体实例}
\label{sec:crontabExamples}
