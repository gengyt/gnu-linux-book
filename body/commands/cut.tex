\section{cut命令的使用}
\label{sec:cutCmd}

cut\index{cut}命令有的时候很有用,比如要获得指定的以某个分隔符分割的列
时,它就可以做到。其实还有比它更强大的工具,如sed及awk等,这里并不介绍
它们,后续章节有介绍。这里就不提及了,跳过它们吧。下面直接看例子吧:

\subsection{常用选项}
\label{subsec:cutOptions}

下面介绍几个常用的cut命令的选项,

\begin{table}[htbp]
  \centering
    \caption{cut常用选项}
    \label{tab:cutSomeOpts}
    \begin{tabular}{cl}
      \toprule
      选项     & 说明 \\
      \midrule
      -c        & 按照字符进行分割 \\
      -d        & 指定分割字段的分隔符,默认是tab \\
      -f        & 指定要显示的列 \\
      \bottomrule
    \end{tabular}
\end{table}

\subsection{一些实例}
\label{subsec:cutInstances}

接下来演示cut命令的一些常用示例,比如我们要查看/etc/passwd文件中用户名,由于用户名是位于/etc/passwd文件中的每一行的第一个以冒号为分割的字段,因此,可以使用cut命令轻松实现取出用户名的需求,

\begin{verbatim}
# cut -d: -f1 /etc/passwd

-d 指定区分列的定界,默认是tab
-f 指定要显示的列
-c 按字符来切割,如
# cut -c1-6 file (取文件第一行的前6个字符)

实验:取IP地址
# ifconfig wlan0 | grep 'inet addr' | cut -d: -f2 | cut -d' ' -f1
# hostname -i[I]

# 显示系统中总内存量
# free |tr -s ' ' |sed '/^Mem/!d' |cut -d" " -f2
\end{verbatim}


\begin{verbatim}
[root@iLiuc ~]# cat test.txt
chuanchuan:goodboy
chuanchuan goodboy

# 以冒号为分割,显示第一列
[root@iLiuc ~]# cut -d: -f1 test.txt
chuanchuan
chuanchuan goodboy

# 以冒号为分割,显示第二列
[root@iLiuc ~]# cut -d: -f2 test.txt
goodboy
chuanchuan goodboy

# 以空格为分割,显示第一列
[root@iLiuc ~]# cut -d" " -f1 test.txt
chuanchuan:goodboy
chuanchuan
\end{verbatim}

我想,知道这么多,应该就可以了。
