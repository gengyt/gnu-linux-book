\section{find命令的使用}

find\index{find}命令很强大,强大到可以写很多东西。这里就介绍如何简单的
使用。直接看例子吧:

\small{
\begin{verbatim}
# linux文件无创建时间
# Access 使用时间  
# Modify 内容修改时间  
# Change 状态改变时间(权限、属主)
# 时间默认以24小时为单位,当前时间到向前24小时为0天,向前48-72小时为2天
# -and 且 匹配两个条件 参数可以确定时间范围 -mtime +2 -and -mtime -4
# -or 或 匹配任意一个条件

# 按文件名查找
find /etc -name http        

# 查找某一类型文件
find . -type f               

# 按照文件权限查找
find / -perm                 

# 按照文件属主查找
find / -user                 

# 按照文件所属的组来查找文件
find / -group                

# 文件使用时间在N天以内
find / -atime -n             

# 文件使用时间在N天以前
find / -atime +n             

# 文件内容改变时间在N天以内
find / -mtime -n             

# 文件内容改变时间在N天以前
find / -mtime +n             

# 文件状态改变时间在N天前
find / -ctime +n             

# 文件状态改变时间在N天内
find / -ctime -n             

# 查找文件长度大于1M字节的文件
find / -size +1000000c -print 

# 按名字查找文件传递给-exec后命令
find /etc -name "passwd*" -exec grep "root" {} \; 

# 查找文件名,不取路径
find . -name 't*' -exec basename {} \;  

# 批量改名(查找err替换为ERR {}文件
find . -type f -name "err*" -exec  rename err ERR {} \; 

# 查找任意一个关键字
find 路径 -name *name1* -or -name *name2* 
\end{verbatim}
}
\normalsize

