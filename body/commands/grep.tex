\section{grep命令的使用}
\label{sec:grepCmd}

grep\index{grep}(global regular expression pattern的缩写)。其实可以把它
理解为过滤关键字用的一个程序。具体怎么用,还是看一个实例吧,然后结束本
节内容。

\subsection{常用选项}

\begin{table}[!htbp]
  \centering
  \caption{grep常用选项}
  \begin{tabular}{l|l}
    \hline
    -A NUM  & 打印出紧随匹配的行之后的下文NUM行 \\
    \hline
    -B NUM  & 打印出匹配的行之前的上文NUM行 \\
    \hline
    -C NUM  & 打印出匹配的行的上下文前后各NUM行 \\
    \hline
    -b      & 在输出的每行前面同时打印出当前行在输入文件中的字节偏移量 \\
    \hline
    -c      & 显示匹配的行数 \\
    \hline
    -f file & 从文件file中获取模式,每行一个 \\
    \hline
    -H      & 为每个匹配的文件打印文件名 \\
    \hline
    -I      & 不搜索二进制文件 \\
    \hline
    -i      & 忽略大小写 \\
    \hline
    -l      & 只显示有匹配的文件的文件名 \\
    \hline
    -L      & 只显示未匹配的文件的文件名 \\
    \hline
    -n      & 输出行号 \\
    \hline
    -o      & 只显示匹配字段 \\
    \hline
    -q      & quiet静默模式 \\
    \hline
    -v      & 只显示不匹配的行 \\
    \hline
  \end{tabular}
\end{table}

\subsection{一些实例}

去掉文件里的注释行和空白行
\begin{verbatim}
# cat filename | grep -v ^$ | grep -v ^# | sudo tee squid.conf
\end{verbatim}

这里我们使用grep命令及cut命令一起把eth0上的IP给取出来,看操作:

\begin{verbatim}
  [root@iLiuc ~]# ifconfig eth0
  eth0 Link encap:Ethernet  HWaddr 5A:B6:4E:85:55:44  
  inet addr:192.168.18.18  Bcast:192.168.18.255  Mask:255.255.255.0
  inet6 addr: fe80::58b6:4eff:fe85:5544/64 Scope:Link
  UP BROADCAST RUNNING MULuTICAST  MTU:1500  Metric:1
  RX packets:11655331 errors:0 dropped:0 overruns:0 frame:0
  TX packets:1074797 errors:0 dropped:0 overruns:0 carrier:0
  collisions:0 txqueuelen:1000 
  RX bytes:4001012028 (3.7 GiB)  TX bytes:628740073 (599.6 MiB)
  Interrupt:185

  # 使用grep命令,匹配关键字,缩小范围
  [root@iLiuc ~]# ifconfig eth0 |grep "inet addr"
  inet addr:192.168.18.18  Bcast:192.168.18.255  Mask:255.255.255.0

  # 使用cut命令,把范围再次缩小些
  [root@iLiuc ~]# ifconfig eth0 |grep "inet addr" | cut -d: -f2
  192.168.18.18  Bcast

  # 是不是快出来了,再使用cut一次,IP地址就出来了
  [root@iLiuc ~]# 自己写出来吧,我已经写得很多了!
\end{verbatim}
