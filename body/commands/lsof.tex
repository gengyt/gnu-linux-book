\section{lsof命令的使用}
\label{sec:lsofCmd}
\index{lsof}

lsof\index{lsof}是“list open files”的缩写,是一个列出当前系统打开文件的
工具。在linux环境下,任何事物都以文件的形式存在,通过文件不仅仅可以访问
常规数据,还可以访问网络连接和硬件。所以如传输控制协议 (TCP) 和用户数据
报协议(UDP) 套接字等,系统在后台都为该应用程序分配了一个文件描述符,无
论这个文件的本质如何,该文件描述符为应用程序与基础操作系统之间的交互提
供了通用接口。因为应用程序打开文件的描述符列表提供了大量关于这个应用程
序本身的信息,因此通过lsof工具能够查看这个列表对系统监测以及排错将是很
有帮助的。

lsof不加任何选项,默认输出所有活动进程打开的文件。

\begin{verbatim}
show all connections with -i

# lsof -i
COMMAND  PID USER   FD   TYPE DEVICE SIZE NODE NAME
dhcpcd  6061 root   4u   IPv4   4510       UDP *:bootpc
sshd    7703 root   3u   IPv6   6499       TCP *:ssh (LISTEN)
sshd    7892 root   3u   IPv6   6757       TCP 10.10.1.5:ssh->192.168.1.5:49901 (ESTABLISHED)

# Get only IPv6 traffic with -i 6
# lsof -i 6

# show only tcp connections (works the same for udp)
# lsof -iTCP

# show networking related to a given port using -i :port
# lsof -i :22

# show connections to a specific host using @host
# lsof -i@172.16.25.18

# show connections based on the host and the port using @host:port
# lsof -i@172.16.25.18:22

# Find listening ports
# find ports that are awaiting connections
# lsof -i -sTCP:LISTEN

# User Information

# We can also get information on various users and what they're doing
# on the system, including their activity on the network, their
# interactions with files, etc.

# show what a given user has open using -u
# lsof -u postfix

# show what all users are doing except a certain user using -u ^user
# lsof -u ^postfix

# Kill everything a given user is doing
# kill -9 `lsof -t -u postfix`
\end{verbatim}

\subsection{恢复删除的文件}

有一次做研发的一位同事,因为系统根目录空间不足,想释放磁盘空间,结果使
用find命令找到了单个文件大于1GB的文件,发现/var/log/messages和
/var/log/warn文件每个都是1.3GB。他的系统根目录空间只有50GB的容量,结果
把/var/log/messages日志文件及/var/log/warn日志文给删除了。删除之后,却
没有达到的目的,磁盘使用空间依然是100\%。

殊不知,当进程打开了某个文件时,只要该进程保持打开该文件,即使将其删除,
它依然存在于磁盘中。这意味着,进程并不知道文件已经被删除,它仍然可以向
打开该文件时提供给它的文件描述符进行读取和写入。除了该进程之外,这个文
件是不可见的,因为已经删除了其相应的目录条目。

拿/var/log/messages文件为例,看看如何在故意删除后如何找回来。我们先看一
下/var/log/messages文件是什么进程打开的。

\begin{verbatim}
[root@iLiuc ~]# lsof /var/log/messages 
COMMAND    PID USER   FD   TYPE DEVICE SIZE/OFF    NODE NAME
syslog-ng 1493 root   10w   REG    8,2 57599903 3080573 /var/log/messages
\end{verbatim}

该命令的输出表明,/var/log/messages文件由syslog-ng进程打开,当前的进程
号为1493,用户身份是root,打开的文件描述符为10且该文件处于只写模式,对
应的TYPE(类型)为REG(常规)文件、磁盘位置、文件大小、索引节点。下面一
一验证:

\begin{verbatim}
[root@iLiuc ~]# stat /var/log/messages 
  File: `/var/log/messages'
  Size: 57603503  	Blocks: 112632     IO Block: 4096   regular file
Device: 802h/2050d	Inode: 3080573     Links: 1
Access: (0640/-rw-r-----)  Uid: (    0/    root)   Gid: (    0/    root)
Access: 2015-03-12 15:33:28.000000000 +0800
Modify: 2015-03-12 16:02:01.000000000 +0800
Change: 2015-03-12 16:02:01.000000000 +0800
 Birth: -
\end{verbatim}

现在,我们可以删除该文件以模拟误删除,

\begin{verbatim}
[root@iLiuc ~]# rm -f /var/log/messages
[root@iLiuc ~]# lsof -n |grep '(deleted)'
syslog-ng 1493  root 10w  REG  8,2 57605375   3080573 /var/log/messages (deleted)
\end{verbatim}

文件已被删除,看怎么恢复吧!从上面的输出信息可以看出之前打开该文件的进
程号及文件描述符,有了这两个信息就足够了。接下来,我们去cat一下/proc目
录中相应的目录中的文件描述符,

\begin{verbatim}
[root@iLiuc ~]# cat /proc/1493/fd/10 |head
Oct  9 03:55:32 linux syslog-ng[5747]: syslog-ng starting up; version='2.0.9'
Oct  9 03:55:32 linux syslog-ng[5747]: syslog-ng starting up; version='2.0.9'
Oct  9 03:55:32 linux syslog-ng[5747]: syslog-ng starting up; version='2.0.9'
Oct  9 03:55:32 linux syslog-ng[5747]: syslog-ng starting up; version='2.0.9'
Oct  9 03:55:32 linux syslog-ng[5747]: syslog-ng starting up; version='2.0.9'
Oct  9 03:55:32 linux syslog-ng[5747]: syslog-ng starting up; version='2.0.9'
Oct  9 03:55:32 linux syslog-ng[5747]: syslog-ng starting up; version='2.0.9'
Oct  9 03:55:32 linux syslog-ng[5747]: syslog-ng starting up; version='2.0.9'
Oct  9 03:55:32 linux syslog-ng[5747]: syslog-ng starting up; version='2.0.9'
Oct  9 03:55:32 linux syslog-ng[5747]: syslog-ng starting up; version='2.0.9'
\end{verbatim}

通过文件描述符查看了相应的数据,那么就可以使用I/O重定向将其复制到文件中,
如cat /proc/1493/fd/10 > /tmp/messages。此时,可以中止该守护进程(这将
  删除 FD,从而删除相应的文件),将这个临时文件复制到所需的位置,然后重
新启动该守护进程。

\begin{verbatim}
[root@iLiuc ~]# cat /proc/1493/fd/10 > /tmp/messages
[root@iLiuc ~]# /etc/init.d/syslog stop
[root@iLiuc ~]# cp /tmp/messages /var/log/messages
[root@iLiuc ~]# /etc/init.d/syslog start
[root@iLiuc ~]# wc -l /var/log/messages 
452113 /var/log/messages

[root@iLiuc ~]# lsof /var/log/messages 
COMMAND    PID USER   FD   TYPE DEVICE SIZE/OFF    NODE NAME
syslog-ng 3819 root    4w   REG    8,2 57608271 3080449 /var/log/messages
\end{verbatim}

我们可以看到,已删除的/var/log/messages文件已回来了!对于许多应用程序,
尤其是日志文件和数据库,这种恢复删除文件的方法非常有用。
