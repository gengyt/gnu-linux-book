
\section{parted命令的使用}
\label{sec:PartedCmd}

Gnu/Linux系统的分区工具通常可以使用fdisk与parted。我们用的比较多的工具
就是fdisk了,这里不介绍它的使用了。这里简单的介绍如何使用parted工具,对
于分区表通常有MBR分区表和GPT分区表对于磁盘大小小于2T的磁盘,我们可以使
用fdisk和parted命令工具进行分区对于MBR分区表的特点(通常使用fdisk命令进
行分区)所支持的最大磁盘大小:2T最多支持4个主分区或者是3个主分区加上一
个扩展分区对于GPT分区表的特点(使用parted命令进行分区)支持最大
卷:18EB(1EB=1024TB)最多支持128个分区

对于parted命令工具分区的介绍

最后,fdisk与parted有些差异。fdisk分区完毕后,需要使用“w”命令才能保存
之前所做的一些操作;而parted则是实时的,每一步操作不需要保存,即时生
效。