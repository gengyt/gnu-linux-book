\section{命令行下的快捷键}
\label{shortcut}

经常在命令行下工作的小伙伴们,可能用的最多的就是两个上下方向键,主要用来调出
历史命令;使用左右箭头使光标向后或向前移动以修改上次使用过的命令。其实
这样做效率并不是很高,有了快捷键可以让我们的效率提高数倍,而且看起来还
更专业、更加Awesome、更加Geek。掌握了这些快捷键,我们可以做到手不离主键
盘区域,完全可以忽略掉键盘上的四个可爱的箭头。当我们熟练之后,会越发喜
欢这种方式。

\subsection{常用快捷键介绍}

下面介绍一些作者在命令行下经常使用的快捷键,这些快捷键在Emacs下面是有同
样的效果的,不信?你可以试试看。其实,Emacs是Gnu/Linux系统下的命令行编
辑器,通过/etc/profile或/etc/bashrc等文件都可以找到相关的设置。

\begin{enumerate}[itemsep=0pt,parsep=0pt]
	\item Ctrl+A\index{Ctrl+A}快捷键
  \begin{quote}
    这里的A可以理解为Head。当我们按下此组合键时,光标就从当前位置移到了
    命令行的起始位置。别只顾着看,动手试试!
  \end{quote}

\item Ctrl+B\index{Ctrl+B}快捷键
  \begin{quote}
    这里的B可以理解为Backward,向后的意思。有时在命令行上,我们把某个命
    令的参数或路径写错了,一般的做法是,使用左箭头,使光标移动到指定的
    位置,然后修改。其实我们完全可以使用Ctrl+B的方式以达到同样的效果。
    别只顾着看,动手试试!
  \end{quote}

\item Ctrl+C\index{Ctrl+C}快捷键
  \begin{quote}
    这个组合键是用来终止当前正在运行的前台进程。在UNIX环境高级编程一书
    上看到了一个用来终止当前运行进程的组合键,是Ctrl+\textbackslash
    \cite{unixenvironment}。别只顾着看,动手试试!
  \end{quote}

\item Ctrl+D\index{Ctrl+D}快捷键
  \begin{quote}
    这个组合键的用途也很广,我主要用此组合键来退出某个程序,如Python、
    MySQL等等。在命令行下意思就不同啦,此时的D可以理解为Delete。按下此
    组合键,会删除当前光标处的字符。别只顾着看,动手试试!
  \end{quote}

\item Ctrl+E\index{Ctrl+E}快捷键
  \begin{quote}
    这里的E可以理解为End。当在命令行按下此组合键时,我们的可爱的光标就
    乖乖地跑到了当前命令行的最后。\marginpar{这是边注一个}
  \end{quote}

\item Ctrl+F\index{Ctrl+F}快捷键
  \begin{quote}
    这里的F可以理解为Forward,向前的意思,等同于按下右箭头。别只顾着看,
    动手试试!
  \end{quote}

\item Ctrl+H\index{Ctrl+H}快捷键
  \begin{quote}
    此组合键相当于键盘上的Backspace键。按下此组合键,它会从当前光标处开
    始向后删除字符。别只顾着看,动手试试!
  \end{quote}

\item Ctrl+J\index{Ctrl+J}快捷键
  \begin{quote}
    此组合键相当于键盘的回车键。按下此组合键,相当于按了一次回车键。在
    Windows的命令行下,Ctrl+M好像是等同于回车键。别只顾看着,动手试试!
  \end{quote}

\item Ctrl+K\index{Ctrl+K}快捷键
  \begin{quote}
    这里的K可以理解为Kill。按下此组合键,会删除从当前光标到本命令行的结
    束的位置的所有字符。别只顾着看,动手试试!
  \end{quote}

\item Ctrl+L\index{Ctrl+L}快捷键
  \begin{quote}
    这里的L可以理解为Clear。按下此组合键相当于执行了clear这条命令,清除
    当前屏幕上的内容。别只顾着看,动手试试!
  \end{quote}

\item Ctrl+N\index{Ctrl+N}快捷键
  \begin{quote}
    这里的N可以理解为Next。这个组合键的作用是用来调出下一条历史命令,与
    之对应的快捷键Ctrl+P是调出上一条历史命令。代替了向下的箭头。别只顾
    着看,动手试试!
  \end{quote}

\item Ctrl+P\index{Ctrl+P}快捷键
  \begin{quote}
    这里的N可以理解为Previous。这个组合键的作用是用来调出上一条历史命令,
    与之对应的快捷键Ctrl+N是调出下一条历史命令。代替了向上的箭头。别只
    顾着看,动手试试!
  \end{quote}

\item Ctrl+R\index{Ctrl+R}快捷键
  \begin{quote}
    这个组合键是用来搜索之前的历史命令。这里的R可以理解为Reverse,反向
    的意思。在Emacs里为向后搜索,与之对应的是Ctrl+S快捷键是向前搜索。不
    过Ctrl+S在命令行里却不是这个作用,而是用来锁屏的。别只顾着看,动手
    试试!
  \end{quote}

\item Ctrl+S\index{Ctrl+S}快捷键
  \begin{quote}
    这个组合键在Emacs里为向后搜索,与之对应的是Ctrl+S快捷键是向前搜索。不
    过Ctrl+S在命令行里却不是这个作用,而是用来锁屏的。别只顾着看,动手
    试试!锁了之后怎么解锁呢?可以是试试Ctrl+Q组合键。
  \end{quote}

\item Ctrl+T\index{Ctrl+T}快捷键
  \begin{quote}
    此组合键是交换两个相邻字符的位置。交换的是当前光标处字符及其当前光
    标前面的字符。比如我们不小心把clear命令写成了clera,此时我们也不用
    把ra两个字符删掉,然后再写上正确的。此时使我们的光标位于字符a上,让
    后按下此组合键,是不是神奇的事情发生了?当然,如果光标在行尾,按下
    此组合键,它会交换光标前的两个连续的字符。在Emacs下面,使用Ctrl+X与
    Ctrl+T两个组合键\footnote{先按下Ctrl+X,然后松开X,继续
      按着Ctrl键,然后再按下T键,即可完成两个组合键的操作。别嫌麻烦,习
      惯就好了。},可以交换当前光标行与上一行的位置。别只顾着看,动手试
    试!
  \end{quote}

\item Ctrl+W\index{Ctrl+W}快捷键
  \begin{quote}
    此组合键在Emacs中的作用是剪切选中区域的文本。在命令行上使用该组合键
    则是往后删除一个字符组合。也就是说,删除光标左边的一个字母组合或单
    词。比如,我们在此命令行上使用了命令如下,“service network
    restart”,让我们的光标位于字符串的restart的后面,按下该组合键,看看
    有何效果?是不是变成“service network”了?确实是这样,如果我们使用
    Backspace键的话,则需要使用7次的按键才能达到一个Ctrl+W的组合键的效
    果。嗯,别只顾着看,动手试试?
  \end{quote}

\item Alt+.\index{Alt+.}快捷键
  \begin{quote}
    此组合键是调出上一条命令的最后一个参数。如上一条命是“service
    network restart”,则“restart”就是最后一个参数。如果我们接下来要敲的
    命令需要用到上一条命令的最后一个参数,则可使用此快捷键,而不需要手
    工输入“restart”了,而且不会出错,节省敲击键盘的次数。如果我们接下来
    想重启httpd服务,则只需要输入“service httpd ”,然后按下“Alt+.”即可
    补全上一条命令的“restart”。在有些终端上,按“Alt+.”组合键可能会没有
    效果,这时可以使用“ESC+.”组合键代替。在Emacs中,ESC键与Alt键是等价
    的。可以动手试试该组合键的效果。
  \end{quote}

\end{enumerate}
