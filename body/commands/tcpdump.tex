\section{tcpdump命令的使用}
\label{sec:tcpdumpCmd}

tcpdump\index{tcpdump}是一款基于命令行的工具,可以通过不同的命令行选项
来改变其状态、捕获数据的数量及捕获数据的方法。tcpdump提供的丰富选项可以
使你很容易的改变程序的运行方式。

下面列举一部分比较常用的选项,

\begin{table}[!htbp]
  \centering
  \caption{tcpdump常用选项}
  \label{tab:tcpdumpOptions}
  \begin{tabular}{ll}
    \toprule
    选项     & 说明 \\
    \midrule
    i     & 指定侦听的网络接口 \\
    v     & 指定详细模式输出详细的报文信息 \\
    vv    & 指定非常详细的模式输出及非常详细的报文信息 \\
    vvv   & 指定更加详细的模式输出及更详细的报文信息 \\
    x     & 规定tcpdump以16进制数格式显示数据包 \\
    X     & 规定tcpdump以hex及ASCII格式显示输出 \\
    XX    & 同上,并显示以太网头部信息 \\
    n     & 在捕获过程中不需要向DNS查询IP地址(显示IP地址及端口号) \\
    F     & 从指定的文件中读取表达式 \\
    D     & 显示tcpdump可以侦听的网络接口列表 \\
    c     & 指定捕获多少数据包,然后停止捕获 \\
    w     & 把捕获到的信息写到一个文件中 \\
    s     & 设置捕获数据包的长度为length \\
    \bottomrule
  \end{tabular}
\end{table}

第一种是关于类型的关键字,主要包括host, net, port,例如host
210.27.38.1,指明210.27.38.1是一台主机,net 202.0.0.0指明202.0.0.0是一个
网络地址,port 23指明端口是23。如果没有指定类型,缺省的类型是host。

\small{
\begin{verbatim}
# 想要截获所有210.27.38.1的主机收到的和发出的所有的数据包:
# tcpdump host 210.27.38.1

# 对本机的udp 123端口进行监视,123为ntp的服务端口
# tcpdump udp port 123

# 如果想要获取主机210.27.38.1接收或发出的telnet包,使用如下命令:
# tcpdump tcp port 23 host 210.27.38.1

# 想要获取主机210.27.38.1和主机210.27.38.2或210.27.38.3的通信,使用命令:
# 在命令行中使用括号时,一定要转义
# tcpdump host 210.27.38.1 and \(210.27.38.2 or 210.27.38.3\)

# 如果想要获取主机210.27.38.1除了和主机210.27.38.2之外所有主机通信的ip包,则:
# tcpdump ip host 210.27.38.1 and ! 210.27.38.2

# tcpdump icmp
# tcpdump port 3306
# tcpdump src port 1025
# tcpdump dst port 389
# tcpdump src port 1025 and tcp
# tcpdump udp and src port 53
\end{verbatim}
}
\normalsize
