\section{yum命令的使用}
\label{sec:yumCmd}
\index{yum}
安装好系统时,在/etc/yum.repos.d目录下回有一个rhel-debuginfo.repo的文件,
我们这里以redhat系统为例进行讲解。不管这个配置文件的名字如何,但文件的
扩展名须为.repo,如redhat.repo也是可以的。我们需要做一些准备工作。

准备系统镜像文件并挂载到本地:

\small{
\begin{verbatim}
[root@iLiuc ~]# ls 
rhel-server-5.5-i386-dvd.iso

[root@iLiuc ~]# mount -o loop rhel-server-5.5-i386-dvd.iso /media
\end{verbatim}
}
\normalsize

复制镜像里的文件到本地目录:

\begin{verbatim}
[root@iLiuc ~]# mkdir /iso
[root@iLiuc ~]# cp -r /media/* /iso
\end{verbatim}

修改这个配置文件:

\begin{verbatim}
[root@iLiuc ~]# cat /etc/yum.repos.d/rhel-debuginfo.repo
[rhel-debuginfo]
name=Red Hat Enterprise Linux $releasever - $basearch - Debug
baseurl=file:///iso/Server
enabled=1
gpgcheck=0
\end{verbatim}

几点说明:

\begin{quote}
    1. [rhel-debuginfo] 中括号里的内容可以随意写 \\
    2. name 这一行可有可无 \\
    3. baseurl 这行要指定我们的资源在哪里 \\
    4. file:// 说明我们使用什么协议,也可以是ftp://等 \\
    5. /iso/Server 指明我们的源在 /iso/Server 目录下
\end{quote}

配置好之后,如何使用呢?直接看操作吧:

\begin{enumerate}[itemsep=0pt,parsep=0pt]
\item 列出我们有哪些yum仓库
  \small{
\begin{verbatim}
[root@iLiuc ~]# yum repolist
\end{verbatim}
  }
  \normalsize

\item 列出仓库里的包
\begin{verbatim}
[root@iLiuc ~]# yum list
Deployment_Guide-en-US.noarch    5.2-11               installed     
GConf2.i386                      2.14.0-9.el5         installed     
ImageMagick.i386                 6.2.8.0-4.el5_1.1    installed     
MAKEDEV.i386                     3.23-1.2             installed     
NetworkManager.i386              1:0.7.0-10.el5       installed     
NetworkManager-glib.i386         1:0.7.0-10.el5       installed     
NetworkManager-gnome.i386        1:0.7.0-10.el5       installed     
ORBit2.i386                      2.14.3-5.el5         installed    
....
yum-utils.noarch                 1.1.16-13.el5        rhel-debuginfo
yum-verify.noarch                1.16-13.el5          rhel-debuginfo
yum-versionlock.noarch           1.1.16-13.el5        rhel-debuginfo
zisofs-tools.i386                1.0.6-3.2.2          rhel-debuginfo
zsh.i386                         4.2.6-3.el5          rhel-debuginfo
zsh-html.i386                    4.2.6-3.el5          rhel-debuginfo
\end{verbatim}
\end{enumerate}

一些说明:
\begin{quote}
    1. 第一列是我们的软件包名 \\
    2. 第二列是对应软件包的版本号 \\
    3. 第三列 \\
    + installed表明该软件包已安装 \\
    + rhel-debuginfo表明包未安装 
\end{quote}

几个常用的yum命令:
\begin{table}[!htbp]
  \centering
    \caption{yum常用命令选项}
    \begin{tabular}{ll}
      \toprule
      命令           & 说明 \\
      \midrule
      repolist       & 列出我们有哪些yum仓库 \\
      list           & 列出仓库里有哪些软件包 \\
      install        & 安装软件包的命令 \\
      groupinstall   & 安装软件包组 \\
      erase          & 移除一个或多个软件包 \\
      whatprovides   & 查询一个命令属于哪个安装包 \\
      \bottomrule
    \end{tabular}
\end{table}

\subsection{一些实例}
\label{sec:yumExamples}