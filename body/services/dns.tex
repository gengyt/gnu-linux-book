\chapter{DNS}
\label{chap:dns}

在TCP/IP网络中,IP地址是网络节点的标识。但是,IP地址是点分十进制数字,
难以记忆。联想到在现实生活中,名字比身份证号码更容易被人记住,那么是否
可以拿名字来标识某个网络节点呢?答案是肯定的。DNS(Domain Name System,
域名系统)是一种用于TCP/IP应用程序的分布式数据库,提供域名与IP地址之间
的转换。

\section{测试环境}
\label{sec:dnsTestEnv}

以下测试是在CentOS6U5 64位系统上进行测试。系统均关闭iptables及selinux,
如果要开启iptables,请设置允许开放本机的53端口。

\begin{table}[htbp]
  \centering
    \caption{DNS演示环境机器一览}
    \label{tab:dnsMachines}
    \begin{tabular}{llr}
      \toprule
      主机名     & IP地址 & 说明 \\
      \midrule
      master01.lavenliu.com  & 192.168.20.134 &  主DNS \\
      minion01.lavenliu.com  & 192.168.20.135 &  辅DNS \\
      minion02.lavenliu.com  & 192.168.20.136 &  客户端 \\
      \bottomrule
    \end{tabular}
\end{table}

\section{安装及配置主DNS}
\label{sec:configMasterDNS}

在master01.lavenliu.com上安装如下软件包,

\begin{verbatim}
[root@master01 ~]# yum install -y bind bind-utils
\end{verbatim}

接下来配置主DNS,有两个文件需要修改,一个是主配置文件/etc/named.conf及/etc/named.rfc1912.zones文件。
首先配置/etc/named.conf文件,配置内容如下,在options字段中做如下修改:

\begin{verbatim}
listen-on port 53 { 127.0.0.1; 192.168.20.134; }; // 192.168.20.134 is Master DNS Server IP
listen-on-v6 port 53 { ::1; };
directory	"/var/named";
dump-file	"/var/named/data/cache_dump.db";
	statistics-file "/var/named/data/named_stats.txt";
	memstatistics-file "/var/named/data/named_mem_stats.txt";
allow-query		{ localhost; 192.168.20.0/24; }; // IP range
allow-transfer	{ localhost; 192.168.20.135; }; // 192.168.20.135 is Slave DNS Server IP
\end{verbatim}

接下来修改/etc/named.rfc1912.zones文件,在该文件末尾追加如下内容:

\begin{verbatim}
cat >> /etc/named.rfc1912.zones <<EOF
zone "lavenliu.com" IN {
     type master;
     file "forward.lavenliu";
     allow-update { none; };
};

zone "20.168.192.in-addr.arpa" IN {
     type master;
     file "reverse.lavenliu";
     allow-update { none; };
};
EOF
\end{verbatim}

由于我们在/etc/named.rfc1912.zones文件中设置了两个区域文件,分别为forward.lavenliu及reverse.lavenliu两个区域文件。
这两个文件分别是正向解析文件与反向解析文件,接下来就创建两个文件,这两个文件的位置默认是/var/named目录下。

首先,创建正向解析文件,

\begin{verbatim}
# vi /var/named/forward.lavenliu
$TTL 86400
@   IN  SOA     master01.lavenliu.com. root.lavenliu.com. (
		        2016051201  ;Serial
				3600        ;Refresh
				1800        ;Retry
				604800      ;Expire
				86400       ;Minimum TTL
)
@           IN  NS          master01.lavenliu.com.
@           IN  NS          minion01.lavenliu.com.
@           IN  A           192.168.20.134
@           IN  A           192.168.20.135
@           IN  A           192.168.20.136
master01    IN  A   	    192.168.20.134
minion01    IN  A   	    192.168.20.135
minion02    IN  A   	    192.168.20.136
\end{verbatim}

其次,创建反向解析文件,

\begin{verbatim}
# vi /var/named/reverse.lavenliu
$TTL 86400
@   IN  SOA     master01.lavenliu.com. root.lavenliu.com. (
		        2016051201  ;Serial
				3600        ;Refresh
				1800        ;Retry
				604800      ;Expire
				86400       ;Minimum TTL
)
@           IN  NS      master01.lavenliu.com.
@           IN  NS      minion01.lavenliu.com.
@           IN  PTR     lavenliu.com.
master01    IN  A   	192.168.20.134
minion01    IN  A   	192.168.20.135
minion02    IN  A   	192.168.20.136
134	    IN	PTR	master01.lavenliu.com.
135	    IN	PTR	minion01.lavenliu.com.
136	    IN	PTR	minion02.lavenliu.com.
\end{verbatim}

\subsection{启动DNS服务}
\label{sec:startDNS}

有了以上步骤的配置,我们就可以启动主DNS了。在启动之前,最好检查一下我们的配置文件的语法是否正确,
另外一个就是文件的权限问题了。由于我们是用root用户创建的正向及反向解析文件,
所以这两个文件的所有者及所属组均为root用户和组,接下来的第一件事情就是修改这两个文件的权限,

\begin{verbatim}
[root@master01 named]# chown -R named.named /var/named
[root@master01 named]# ll /var/named/
total 36
drwxrwx--- 2 named named 4096 Mar 16 21:25 data
drwxrwx--- 2 named named 4096 Mar 16 21:25 dynamic
-rw-r--r-- 1 named named  531 May 14 21:54 forward.lavenliu
-rw-r----- 1 named named 2075 Apr 23  2014 named.ca
-rw-r----- 1 named named  152 Dec 15  2009 named.empty
-rw-r----- 1 named named  152 Jun 21  2007 named.localhost
-rw-r----- 1 named named  168 Dec 15  2009 named.loopback
-rw-r--r-- 1 named named  564 May 14 21:57 reverse.lavenliu
drwxrwx--- 2 named named 4096 Mar 16 21:25 slaves
\end{verbatim}

接下来检查正向及反向解析文件的语法正确性,

\begin{verbatim}
[root@master01 ~]# named-checkconf /etc/named.conf
[root@master01 ~]# named-checkzone lavenliu.com /var/named/forward.lavenliu
zone lavenliu.com/IN: loaded serial 2016051201
OK
[root@master01 ~]# named-checkzone lavenliu.com /var/named/reverse.lavenliu
zone lavenliu.com/IN: loaded serial 2016051201
OK
\end{verbatim}

看上去没有问题,接下来就可以启动named服务了,
\begin{verbatim}
[root@master01 ~]# /etc/init.d/named start 
\end{verbatim}

启动之后,最好检查一下named的进程是否启动成功及53端口是否监听,

\begin{verbatim}
ps -ef |grep named |grep -v grep
netstat -antup |grep named |grep -v grep
\end{verbatim}

%%% Local Variables:
%%% mode: latex
%%% TeX-master: t
%%% End:
