\part{附录部分}
\chapter{书中的源码}
\label{chap:sourceCode}

小白在写书过程中,书中用到了很多插图及少量的Bash脚本。这些插图都是小白
使用代码给绘制出来的。这里小白把自己所使用的工具及插图的源代码在本章里
给出。有需要的同志,可以随意修改,成为自己所需的图形。脚本的代码这里不
再给出。

为什么要使用这么古老的绘图工具?首先,小白使用的Ubuntu系统,上面没有安
装Office套件。这种图形看起来很专业\footnote{科技之类的文章好像都是这种
  形式};其次,绘制出的图形很精确\footnote{指哪画哪}。然而,它也有很大
缺点。首先,不可视化,很多人用不习惯(多用几次就好了)\footnote{编译之
  后才能看到效果};其次,不适合给领导做汇报用\footnote{使用PPT的方式或
  许更好,PPT越花哨越好}。

\section{第1章插图源码}

图\ref{fig:UnixTopo}的源代码,使用的绘图工具为metapost。如何编译生
成PDF图形呢?

\begin{verbatim}
# mpost -tex=latex source.mp
\end{verbatim}

\noindent 源码如下:
\begin{verbatim}
verbatimtex
\documentclass[10pt]{article}
\usepackage{CJK}
\AtBeginDocument{\begin{CJK}{UTF8}{gbsn}}
\AtEndDocument{\end{CJK}}
\begin{document}
etex
  
input boxes;
input rboxes;

% system variables
ahangle := 40;

% fig0 is linux virtual file system topo
beginfig(0);
  boxit.a(btex Shell解释程序 etex);
  boxit.b(btex \begin{tabular}{c}
      用户程序\quad各种应用程序包 \\
      系统命令\quad窗口命令\quad库函数
      \end{tabular} etex);
  boxit.c(btex 系统调用 etex);
  boxit.d(btex \begin{tabular}{c}
      核心层: \\
      存储管理\quad进程管理 \\
      设备管理\quad文件管理
      \end{tabular} etex);
  boxit.e(btex 硬件层 etex);
  a.sw = b.nw; a.se = b.ne;
  b.sw = c.nw; b.se = c.ne;
  c.sw = d.nw; c.se = d.ne;
  d.sw = e.nw; d.se = e.ne;
  a.se - a.sw = (140,0);
  b.se - b.sw = (140,0);
  c.se - c.sw = (140,0);
  d.se - d.sw = (140,0);
  e.se - e.sw = (140,0);

  circleit.cc(btex 用户 etex);
  circleit.dd(btex 用户 etex);

  cc.dx=cc.dy; dd.dx=dd.dy;

  cc.w = a.nw + (5,30);
  dd.e = a.ne - (5,-30);
  
  drawboxed(a,b,c,d,cc,dd);

  pair A[];
  A[1] = 1/3[a.nw,a.ne];
  A[2] = 2/3[a.nw,a.ne];
  draw cc.s -- A[1];
  draw dd.s -- A[2];

endfig;
end
\end{verbatim}

\section{第2章插图源码}

Libvirt拓扑图,

\begin{verbatim}
.PS
h = .1
dh = .02
dw = .1
[
	Userspacetools: [
		boxht = 2.5*h; boxwid = 8*dw; boxrad = dh
		movewid = 2*dh
		A: box "virsh"; move
		B: box "virt-manager"; move
		C: box "virt-viewer"; move
		D: box "virt-install"; move
		E: box "other tools"
	]
	Userspace: [
		boxht = 7*h; boxwid = 50*dw; boxrad = 2*dh
		AA: box
	] with .c at Userspacetools.c + (0,h/1.8)
	F: "Userspace Management Tools" at last [].n - (0,h+2*dh)

Libvirt: box ht 4*h wid 25*dw "Libvirt" "(Libvirt API)" with .n at last [].s - (0,3*h)

Hypervisoroutline: [
	Virtualtype: [
		boxht = 2*h; boxwid = 12*dw; boxrad = dh
		movewid = 2*dh
		A: box "VMware"; move
		B: box "Xen"; move
		C: box "KVM"; move
		D: box "Hyper-V"
	]
	Hypervisor: [
		boxht = 5*h; boxwid = 50*dw; boxrad = 2*dh
		AA: box
	] with .c at Virtualtype.c + (0,h/1.8)
	F: "Hypervisor Layer" at last [].n - (0,h+2*dh)
] with .n at Libvirt.s - (0,3*h)

XXX: [
	VMwareoutline: [
		VMware: [
			boxht = 3.5*h; boxwid = 5*dw; boxrad = dh
			movewid = 2*dh
			VM1: box "Guest 1"; move
			VM2: box "Guest 2"
		] with .n at Hypervisoroutline.Virtualtype.A.s - (0,3*h)
		box dashed ht last [].ht+dw wid last [].wid+dw at last []
	] 
	
	move 5*dh
	
	Xenoutline: [
		Xen: [
			 boxht = 3.5*h; boxwid = 5*dw; boxrad = dh
			 movewid = 2*dh
			 VM1: box "Dom0" "Guest"; move
			 VM2: box "DomU" "Guest"
		]
		box dashed ht last [].ht+dw wid last [].wid+dw at last []
	]

	move 5*dh

	Kvmoutline: [
		Kvm: [
			 boxht = 1.75*h; boxwid = 5*dw; boxrad = dh
			 movewid = 2*dh
			 VM1: [
			 	  Qemu1: box "Qemu"
			 	  Guest01: box "Guest 1" with .n at Qemu1.s
			 ]
			 
			 move
			 
			 VM2: [
			 	  Qemu1: box "Qemu"
				  Guest01: box "Guest 2" with .n at Qemu1.s
			 ]
		]
		box dashed ht last [].ht+dw wid last [].wid+dw at last []
	]
	
	move 5*dh

	Hypervoutline: [
		Hyperv: [
			boxht = 3.5*h; boxwid = 5*dw; boxrad = dh
			movewid = 2*dh
			VM1: box "Guest 1"; move
			VM2: box "Guest 2"
		]
		box dashed ht last [].ht+dw wid last [].wid+dw at last []
	]
] with .n at last [].s - (0,3*h)

arrow from Userspacetools.A.s to Libvirt.nw
arrow from Userspacetools.B.s to 1/2 <Libvirt.nw,Libvirt.n>
arrow from Userspacetools.C.s to Libvirt.n
arrow from Userspacetools.D.s to 1/2 <Libvirt.n,Libvirt.ne>
arrow from Userspacetools.E.s to Libvirt.ne

arrow from Libvirt.s to 3rd [].Hypervisor.n

arrow from Hypervisoroutline.Virtualtype.A.s to XXX.VMwareoutline.VMware.n
arrow from Hypervisoroutline.Virtualtype.B.s to XXX.Xenoutline.Xen.n
arrow from Hypervisoroutline.Virtualtype.C.s to XXX.Kvmoutline.Kvm.n
arrow from Hypervisoroutline.Virtualtype.D.s to XXX.Hypervoutline.Hyperv.n

]
.PE
\end{verbatim}

libvirt与node与hypervisor,

\begin{verbatim}
.PS
[
		A: [
		   box dashed wid 0.7 ht 0.75 rad 0.05 "Domain 1"
		   move 0.2
		   box dashed wid 0.7 ht 0.75 rad 0.05 "Domain 2"
		   move same
		   box dashed wid 0.7 ht 0.75 rad 0.05 "Domain N"
		]

		B: [
		   box wid 2.5 ht 0.2 rad 0.1 "Hypervisor"
  		] with .n at A .s - (0,0.15)
]

box ht last [].ht+0.45 wid last [].wid+0.3 at last [] + (0,0.05)
"Node" at last box .n - (0,0.15)
.PE
\end{verbatim}

KVM桥接网络,

\begin{verbatim}
.PS
[
		A: [
		   box dashed wid 0.7 ht 0.75 rad 0.05 "Domain 1"
		   move 0.2
		   box dashed wid 0.7 ht 0.75 rad 0.05 "Domain 2"
		   move same
		   box dashed wid 0.7 ht 0.75 rad 0.05 "Domain N"
		]

		B: [
		   box wid 2.5 ht 0.2 rad 0.1 "Hypervisor"
  		] with .n at A .s - (0,0.15)
]

box ht last [].ht+0.45 wid last [].wid+0.3 at last [] + (0,0.05)
"Node" at last box .n - (0,0.15)
.PE
\end{verbatim}

saltstack通讯模型,

\begin{verbatim}
.PS
scale = 2.54

define port { [
       Port: box wid $1 ht $2 $3 "$4"
       ]
}

Master: box wid 5.5 ht 2 rad 0.2;

"master" at Master.n below;

Port4505: [ port(2,1,"Publish Port",4505) ] with .se at Master.s - (0.25,0);
Port4506: [ port(2,1,"Return Port",4506) ] with .sw at Master.s + (0.25,0);

Minion1: [ box wid 1.5 ht 1 rad 0.1 "minion1" ] with .nw at Master.sw - (0,1.5);
Minion2: [ box wid 1.5 ht 1 rad 0.1 "minion2" ] with .ne at Master.se - (0,1.5);

line -> from 1/3 <Port4505.sw,Port4505.se> to 1/3 <Minion1.nw,Minion1.ne>;
line -> from 1/3 <Port4505.se,Port4505.sw> to 1/3 <Minion2.nw,Minion2.ne>;
line -> from 2/3 <Minion1.nw,Minion1.ne> to 1/3 <Port4506.sw,Port4506.se> dashed;
line -> from 2/3 <Minion2.nw,Minion2.ne> to 2/3 <Port4506.sw,Port4506.se> dashed;
.PE
\end{verbatim}

ELK拓扑图,

\begin{verbatim}
.PS

define shipper {
	box wid $1 ht $2 rad $3 "Shipper" "LogStash";
}

[
	shipper(0.85,0.4,0.02);
]

[
	shipper(0.85,0.4,0.02);
] with .n at last [].s - (0,0.35);

[
	shipper(0.85,0.4,0.02);
] with .n at last [].s - (0,0.35);

Broker: circle rad 0.5 "Borker" "Redis" "192.168.20.138" with .w at 2nd [].e + (0.35,0);

move right 0.35;

Indexer: box wid 1.15 ht 0.55 rad 0.02 "Indexer" "LogStash" "192.168.20.139";

move same;

Elastic: box wid 1.15 ht 0.75 rad 0.02 "Search &" "Storage" "ElasticSearch" "192.168.20.139";

move same;

Kibana: box wid 1.15 ht 1.9 rad 0.02 "Web" "Interface" "LogStash" "192.168.20.139";

point01 = 1st [].e.y;
Line1: line right from 1st [].e to (Broker.n.x,point01);
line -> from Line1.end to Broker.n;

point02 = last [].e.y;
Line2: line right from last [].e to (Broker.s.x,point02);
line -> from Line2.end to Broker.s;

line -> from 2nd [] .e to Broker.w;
line -> from Broker.e to Indexer.w;
line -> from Indexer.e to Elastic.w;
line -> from Kibana.w to Elastic.e;
.PE
\end{verbatim}

AWK工作流程图,

\begin{verbatim}
.PS
A: box ht 0.8 rad 0.08 "Line 1" "Line 2" "Line 3" "Line 4" "..."
INFILE: "\fBInput file\fP" with .n at A.s - (0,0.15)
B: box width 1.8 "Read a Line" with .nw at A.ne + (0.5,0)
C: box width 1.8 "Execute awk commands" "in the \fBbody\fP block on the line" at B - (0,0.8)
D: box width 1.8 "\fBRepeat\fP for the next line until" "end of the input file" at C - (0,0.8)
E: box width 1.8 rad 0.1 "Execute awk commands in" "the \fBEND\fP Block" at D - (0,0.8)
F: box width 1.8 "Execute awk commands in" "the \fBBEGIN\fP Block" at B + (0,0.8)

L1: line chop 0.01 chop 0.9 from 1st box at 1/3 <A.e,A.ne> to B ->

L2: line down from B to C -> chop
L3: arrow down from C to D chop
L4: arrow down from D to E chop

L5: line down 2 from L1 .center
L6: line right from L5.end to L4.center
L7: arrow  from F.s to B.n
.PE
\end{verbatim}

SED工作流程图,

\begin{verbatim}
.PS
A: box ht 0.8 rad 0.08 "Line 1" "Line 2" "Line 3" "Line 4" "..."
INFILE: "\fBInput file\fP" with .n at A.s - (0,0.15)
B: box width 1.8 "\fBRead\fP a line into the" "pattern space" with .nw at A.ne + (0.5,0)
C: box width 1.8 "\fBExecute\fP given sed commands" "on the pattern space" at B - (0,0.8)
D: box width 1.8 "\fBPrint\fP the pattern space" "and empty it" at C - (0,0.8)
E: box width 1.8 "\fBRepeat\fP for the next line until" "end of the input file" at D - (0,0.8)
F: box width 0.75 ht 0.35 rad 0.05 "Line 1" with .w at B.e + (0.35,0)
PATTERN: "\fBPattern space\fP" with .n at F.s - (0,0.15)
G: circle "Output" with .w at D.e + (0.35,0)

L1: line chop 0.01 chop 0.9 from 1st box at 1/3 <A.e,A.ne> to B ->

L2: line down from B to C -> chop
L3: arrow down from C to D chop
L4: arrow down from D to E chop

L5: line dashed from B.e to F.w ->
L6: line dashed from C.e to F.w ->
L7: line dashed from D.e to G.w ->

L8: line down 2.85 from L1 .center to E.sw - (0.2,0.2)
L9: line right from L8.end to E.s - (0,0.2)
L10: line from E.s to L9.end
.PE
\end{verbatim}

Linux模型,

\begin{verbatim}
verbatimtex
\documentclass[10pt]{article}
\usepackage{CJK}
\AtBeginDocument{\begin{CJK}{UTF8}{gbsn}}
\AtEndDocument{\end{CJK}}
\begin{document}
etex
  
input boxes;
input rboxes;

% system variables
ahangle := 40;

% fig0 is linux virtual file system topo
outputtemplate := "vfs.mps";
beginfig(0);
  defaultfont:="ptmr8r";
  boxit.a(btex 用户进程 etex);
  boxit.b(btex 系统调用界面 etex);
  boxit.c(btex VFS etex);
  boxit.d(btex Ext3 etex);
  boxit.e(btex Buffer Cache etex);
  boxit.f(btex 设备驱动程序 etex);
  boxit.g(btex 硬盘控制器 etex);
  boxit.minix(btex Minix etex);
  boxit.nfs(btex NFS etex);
  boxit.sysv(btex Sysv etex);
  boxit.direc(btex 目录cache etex);
  boxit.inode(btex Inode cache etex);
  boxit.hard(btex etex);

  % Len is the box's length
  % Hig is the box's hight
  numeric Len;
  numeric Hig;
  Len := 65;
  Hig := 14pt;
  a.e - a.w = (Len,0); a.n - a.s = (0,Hig);
  b.e - b.w = (Len,0); b.n - b.s = (0,Hig);
  c.e - c.w = (Len,0); c.n - c.s = (0,Hig);
  d.e - d.w = (35,0); d.n - d.s = (0,Hig);
  minix.e - minix.w = (35,0); minix.n - minix.s = (0,Hig);
  nfs.e - nfs.w = (35,0); nfs.n - nfs.s = (0,Hig);
  sysv.e - sysv.w = (35,0); sysv.n - sysv.s = (0,Hig);
  e.e - e.w = (Len,0); e.n - e.s = (0,Hig);
  f.e - f.w = (Len,0); f.n - f.s = (0,Hig);
  g.e - g.w = (Len,0); g.n - g.s = (0,Hig);
  direc.e - direc.w = (Len,0); direc.n - direc.s = (0,Hig);
  inode.e - inode.w = (Len,0); inode.n - inode.s = (0,Hig);

  % Dis is the hight between the boxes
  numeric Dis;
  Dis := 20;
  a.s - b.n = (0,30);
  b.s - c.n = (0,Dis);
  c.s - d.ne = (5,Dis);
  d.se - e.n = (-5,Dis);
  c.s - nfs.nw = (-5,Dis);
  d.w - minix.e = (10,0);
  sysv.w - nfs.e = (10,0);
  e.s - f.n = (0,Dis);
  f.s - g.n = (0,30);
  c.w - direc.e = (Dis,0);
  c.e - inode.w = (-Dis,0);
  hard.c = g.c;
  hard.e - hard.w = (100,0);
  hard.n - hard.s = (0,34);
  drawboxed(a,b,c,d,e,f,g,minix,nfs,sysv,direc,inode,hard);

  % draw the connectors
  drawarrow a.s -- b.n;
  drawarrow b.s -- c.n;
  
  drawarrow c.s -- minix.n;
  drawarrow c.s -- d.n;
  drawarrow c.s -- nfs.n;
  drawarrow c.s -- sysv.n;

  pair A[];
  A[1] = 1/5[e.nw,e.ne];
  A[2] = 2/5[e.nw,e.ne];
  A[3] = 3/5[e.nw,e.ne];
  A[4] = 4/5[e.nw,e.ne];
  drawarrow minix.s -- A[1];
  drawarrow d.s -- A[2];
  drawarrow nfs.s -- A[3];
  drawarrow sysv.s -- A[4];

  drawarrow e.s -- f.n;
  drawarrow f.s -- g.n;

  drawarrow c.w -- direc.e;
  drawarrow c.e -- inode.w;

  % draw the outline
  pair B[];
  B[1] = direc.w - (5,-56);
  B[2] = inode.e - (-5,-56);
  B[3] = inode.e - (-5,119);
  B[4] = direc.w - (5,119);

  draw B[1] -- B[2] -- B[3] -- B[4] -- cycle dashed evenly;
  label.rt(btex 硬件层 etex,hard.e);
  label.rt(btex Linux内核层 etex,b.e+(15,0));
  label.rt(btex 系统调用 etex,a.se+(15,-5));
endfig;

outputtemplate := "os-arch.mps";
beginfig(1);
  boxjoin(a.nw=b.sw; a.ne=b.se);
  boxit.a(btex Device Driver etex);
  boxit.b();
  boxit.c();
  boxit.d();
  boxit.e();

  numeric Len[],Hig[];
  Len[1] := 300;
  Hig[1] := 15;
  Hig[2] := 30;
  a.e - a.w = (Len[1],0); a.n - a.s = (0,Hig[1]);
  b.n - b.s = (0,Hig[2]); c.n - c.s = (0,Hig[2]);
  d.n - d.s = (0,Hig[1]); e.n - e.s = (0,Hig[2]);
  drawboxed(a,b,c,d,e);

endfig;

outputtemplate := "osarch.mps";
beginfig(2);
  boxjoin(a.nw=b.sw; a.ne=b.se);
  boxit.a(btex Block Device Driver etex);
  boxit.b(btex Volume Manager etex);
  boxit.c(btex File Systems etex);
  boxit.d(btex VFS etex);

  boxjoin(e.nw=f.sw; e.ne=f.se);
  boxit.e(btex Ethernet etex);
  boxit.f(btex IP etex);
  boxit.g(btex TCP/UDP etex);
  boxit.h(btex Socket etex);

  e.nw = a.ne; e.sw = a.se;  
  numeric Len[],Hig[];
  Len[1] := 95;
  Hig[1] := 15;
  Hig[2] := 30;
  a.e - a.w = (Len[1],0); a.n - a.s = (0,Hig[1]);
  b.n - b.s = (0,Hig[1]); c.n - c.s = (0,Hig[1]);
  d.n - d.s = (0,Hig[1]); e.n - e.s = (0,Hig[1]);
  f.n - f.s = (0,Hig[1]); g.n - g.s = (0,Hig[1]);

  drawboxed(a,b,c,d,e,f,g,h);
endfig;
end
\end{verbatim}